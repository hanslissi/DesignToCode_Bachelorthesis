\newpage
\section{Introduction}
The motivation for this thesis stems from the author's current studies in Information Design and
professional experience working as both a UI/UX designer and web-developer. Through this unique and
holistic perspective the author realized how challenging and complex the process of designing
digital products really is. 

While designers are more focused on aesthetics, usability and user needs, developers mostly center
on structure, performance and logic. These fundamentally different priorities can create a sense of
disconnect. Not knowing what the other discipline's needs, processes and constraints are, they often
end up working in silos. This can result in demoralized teams and leads to miscommunication,
interface inconsistencies and digital experiences that feel unfinished.

However, both disciplines have methodologies and skills that complement each other. The author
noticed that when collaboration works well, truly remarkable ideas come to fruition and great
products are crafted. That is why this thesis focuses on answering the following research question:

"How can UI designers adapt their workflows and design systems to meet the needs of developers in
small, iterative teams working with component-based web development, specifically by addressing
requirements for semantic structure, maintainability, and efficient design-to-code handoff?"

This is done by first exploring the fundamentals of web design, including usability principles and
design system approaches like Atomic Design. The thesis continues by comparing traditional to modern
project management, to understand which model offers the best collaboration between disciplines. In
Chapter 4, insights from a cross-disciplinary survey are analyzed, revealing the most common pain
points in the handoff process. Building on these findings and further user research, a practical
solution to this design-to-code problem, called The DesignAPI, is developed. 
