\newpage
\section{Abstract}
In recent years, web-development has become a fast-paced landscape. For high-quality digital
products, close collaboration between UI design and development is essential. In practice, however,
misunderstandings and inefficiencies arise, leading to a worse user experience and demoralized
teams.

This bachelor's thesis explores how design workflows and practices can be adapted to better align
with the needs of developers working in small, iterative teams. Based on a strong theoretical
foundation and qualitative insights from surveys and interviews, the thesis presents a practical
educational solution. Its goal is to improve design-to-code workflows by building
shared-understanding, establishing common principles and enhancing communication. 

\section{Kurzfassung}
In den letzten Jahren hat sich die Web-Entwicklung zu einem schnelllebigen Feld entwickelt. Für
hochwertige digitale Produkte müssen UI-Design und Entwicklung eng zusammenarbeiten. In der Praxis
führen jedoch Missverständnisse und ineffiziente Prozesse häufig zu einer verschlechterten User
Experience und demotivierten Teams.

Diese Bachelorarbeit untersucht, wie Design-Prozesse und Methoden so angepasst werden können, dass
sie den Anforderungen von Entwickler:innen in kleinen, iterativen Teams besser gerecht werden.
Basierend auf einer soliden theoretischen Grundlage sowie qualitativen Erkenntnissen aus Umfragen
und Interviews wird eine lehrreiche und praxisorientierte Lösung vorgestellt. Ziel ist es,
Design-to-Code-Prozesse durch ein gemeinsames Verständnis, klare Prinzipien und die Förderung von
Kommunikation nachhaltig zu verbessern.
