\newpage
\subsection{Survey}

\subsubsection{Goal of the Survey}
The goal of this qualitative survey was to gather in-depth data about designer-developer
collaboration patterns and identify key friction points in their workflows.

\subsubsection{Survey Design}
Although the survey was designed to be answered by both designers and developers, the questions were
tailored to their fields. After an initial question about their role, the survey branched into two
paths. One for designers and one for developers. While the structure and order of the questions was
similar, the wording and focus of some was slighlty adjusted. The thought behind that was to make it
easy for both parties to answer, while being able to compare and align the answers later on.

Examples of questions include: % NOTE show them in a better way than a table...
\begin{center}
    \begin{tabular}{|m{7.5cm}|m{7.5cm}|}
        \hline
        \textbf{Designer Question}                                                                         & \textbf{Developer Question}                                                                                \\
        \hline
        How would you rate your understanding of web development?                                          & How comfortable are you working with design tools like Figma?                                              \\
        \hline
        What do you think were the main reasons for these discrepancies between design and implementation? & What issues do you commonly face with design handoffs and implementing designs?                            \\
        \hline
        When making a new version [\dots], how do you communicate these changes to the developers?         & When a component needs to be changed, [\dots] how do you expect the designer to communicate these changes? \\
        \hline
    \end{tabular}
\end{center}

Formulating the questions in this way allowed for seeing if there are any discrepancies in the
workflow and how expectations align. The majority of questions were multiple choice, with some
open-ended ones to allow for some more reflective answers.

As the survey was designed to reach both beginners and experienced professionals, the survey was
distributed to a wide range of channels:
\begin{itemize}
    \item Two small web development companies that focus on design (Webduett and Userbrain)
    \item A list of Design and Software Development students of FH Joanneum
    \item LinkedIn connections
    \item Personal designer and developer contacts
\end{itemize}

% NOTE: Maybe show the graphics specially made for the survey and mention humor that was built in

\subsubsection{Presentation of Survey Results}
Over a duration of 30 days, 22 people responded to the survey. 11 of which considered themselves
primarily as designers, 8 as developers and 3 were not in the target group. While the sample size is
small, the results are qualitative and provide valuable insights and perspectives.

The majority of designers have not yet had as many years of professional experience as the
developers.  % NOTE: 
% NOTE: Show some numbers here, bar graphs with two bars (one for designer one for developer could
% be suitable here)

The survey supported online claims of Figma being the most popular design tool. 90\% of designers
said they use Figma for designing interfaces. Tools like Photoshop are still used by ~30\%.

When asked about their understanding of the other discipline, the majority of designers rated their
web development knowledge at 3/5 stars, knowing the basics of HTML, CSS and JavaScript. 2 of
them even rated it as "5/5 stars - Honestly could do the developers job :)". Also Developers seem to
know their way around design tools like Figma with 75\% rating their comfort level as 3/5 or higher.
Responses suggest that while there is a general understanding of the other disciplines tools, there
is room for improvement.

81\% of designers said they have experienced discrepancies between design and implementation. The
most common reasons being time constraints, misinterpretation by developers and "Special cases like
complex animations, Popups, Microinteractions". When asked about the main issues with design
handoffs, developers claimed that "Too few designed states (missing hover, focus, \dots)" and
the unclarity of responsive layouts were the most common. Noteworthy is also that devs think complex
components as well as animations, popups or modals are often not well enough specified or
documented. Furthermore, designs not having a component-based approach seem to be a problem, despite
over 80\% of designers claiming they work with such an approach.
These results suggest that there is a common ground for persisting issues. While designers seem to
be aware of component-based design, there may be a misalignment in how it is understood between
disciplines.
% NOTE: In discussion mention that maybe the designers should learn component based design a little
% better or that they should learn how to structure them better

Despite none of the engineers thinking that "lack of communication with designers" is a re-occuring
issue, some designers state that "more time together with developer[s] and clear communication"
would be beneficial for the workflow.

Regarding practices that have improved the designer-developer collaboration, the most frequent
mentioned factors were proper documentation, design systems, well-structured components and early
involvment of developers. One developer noted that best results were achieved when the disciplines
tried to work out solutions together:
"Implementing a collaborative approach early on in projects: Continuous check-ins with designers and
developers to find the best solution within the given target, budget, and timeline."

\textbf{Bad and Good Practices:}\\
Respondents identified several specific bad practices that they would like to see less of in the
future. Here, the most commonly named was that there are too many different components and variants,
with one dev recognizing that maybe designers and developers have a different understanding of what
a component is: "Having similar components in thousands of variants is the default mode in design
tools, while in development having few components with a few variants is the default". Also
"inconsistency thoughout repearing patterns" were mentioned as a common practice as well as not
using the already designed or developed components.
These results shows again that a diverging understanding of components may be present and that
design systems are not yet used to their full potential.

On the other hand, helpful practices mentioned by the engineers include using clearly defined design
tokens, fonts and colors consistently as well as defining edge-case behaviours and annotating
components. \\

Building on these findings, the next section discusses the underlying causes and possible solutions
to these problems.

\subsubsection{Discussion \& Conclusion}