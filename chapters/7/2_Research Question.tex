\newpage
\subsection{Research Question}
"How can UI designers adapt their workflows and design systems to meet the needs of developers in
small, iterative teams working with component-based web development, specifically by addressing
requirements for semantic structure, maintainability, and efficient design-to-code handoff?"

Having done research and completing the practical part, this important and complex research question
is ready to be answered.

\textbf{Embrace and align on component-based thinking}
As developers use a component-based approach when working with modern frontend frameworks like React
or Vue, designers should structure their UI elements following this approach. Methodologies like
Atomic Design can be used to hierarchically build up UIs. Although designers claim to use this
approach already, their understanding often diverges from developers. That is why it is crucial to
align the mental models and define shared principles and nomenclature for components, their states
and variants to reduce friction early.

\textbf{Build maintainable, semantic design systems}
The design system is the centre of interaction between the different disciplines. Design and code
parity can be improved by having:
\begin{itemize}
    \item Consistent shared naming conventions
    \item Structured components/variants that try to mimic code structure
    \item (Semantic) design tokens for colours, spacing values and typography
\end{itemize}

\textbf{Share tools and communicate early}
Make sure developers are comfortable using design tools like Figma as they provide them with the
necessary features to take advantage of a strongly built design system. Communicate where assets are
located and align on which responsive breakpoint values should be used. To counter time constraints
ask engineers what parts of a component are important to be documented and which edge cases could
cause problems in the implementation phase. This way designers can be sure to have set the right
priorities and developers are sure to have all necessary information for implementation.

\textbf{Continuous collaboration and shared understanding}
Include Lean UX practices, such as informal collaborative design meetings or brainstorming sessions
to make sure design and development never work in silos, but collaborate to make the best possible
products. Lastly, push developers to learn about Figma and motivate designers to understand the
basics  of web development like HTML, CSS and JavaScript.

\textbf{The DesignAPI}
The DesignAPI's aim is to help with all the topics mentioned above and is built from these research
results and insights. It was created to educate both parties, to build a shared understanding for the
most challenging topics and to boost casual collaboration between design and development.
