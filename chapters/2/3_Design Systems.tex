\newpage
\subsection{Atomic Design Systems}
So, let us take a closer look at Atomic Design and the reason why I believe it is the best fitting
design system methology for Design and Code collaboration. Let's begin by looking at how the web
designer Brad Frost defined hierarchical parts of the system, by looking at chemistry. \\\\
\textbf{Atoms} \\
Atoms are the smallest building blocks of the design. They are the basic elements like buttons,
textfields, colors, fonts or icons. Things that cannot be broken down any further. \\\\
\textbf{Molecules} \\
Molecules are groups of atoms that together form a simple group of UI. For example, combining
a headline, an icon and filter buttons it would form a neat filter bar. \\\\
\textbf{Organisms} \\
Organisms are more complex groups of UI elements. They can be a combination of atoms, molecules and
also other organisms. When combining the filter bar molecule, a few buttons and a blog post grid
organism, it would form a functional blog post overview organism. \\\\
\textbf{Templates} \\
Templates are like blueprints of pages. They consist of organisms and molecules and define the
layout of a page, giving actual meaning to the UI elements. \\\\
\textbf{Pages}\\
Pages take templates as a basis and fill them with actual content. These represent how the final
delivered product will or should look like. \\
\vglcite[43,53]{frostAtomicDesign2016}\\

I like to picture this methodology as building with Lego. Atoms are the individual Lego bricks
which you can't make any smaller. From there, you start sticking them together,
creating bigger and more exciting builds. Then, by using these defined pieces, you can construct an entire
Lego world that looks consistent but still gives you the freedom to build anything you imagine.

I might have gotten a bit carried away with the Lego analogy, but it is exactly what we want from
design systems. Having a set of clearly defined rules, which give both designers and engineers the
time to actually focus on solving user needs without having to worry about the design getting
inconsistent. \vglcite[13]{vesselovBuildingDesignSystems2019}


\textbf{Change it once, change it everywhere}\\

\textbf{Consistency}\\

\textbf{Adhere to the Single Responsibility Principle}\\

\textbf{Support iterative workflows}\\

And many more.
