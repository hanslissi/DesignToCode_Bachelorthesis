\newpage
\subsection{Component Based UI Design}
UI Designs without realising them through code are similar to having a blueprint for a building.
They provide a detailed plan of how the final product should look like and function, but remain
mostly nonfunctional on their own. Similarly, engineers rely on these blueprints to bring the vision
to life. It is the collaboration of the two disciplines that result in a building, or in this case,
a website. UI Design depends on development and vice versa. 

Because of this strong dependency, it only makes sense that both disciplines share common concepts
and principles. One of these concepts is called Component Based Design.

\subsubsection{The Concept of Components}
This concept originates from a programming paradigm called Component Based Software Engineering
(CBSE). Without getting too much into the technical details, CBSE is about breaking down software
into smaller, independent and reusable parts called components. This modularity allows for easier
maintenance, scalability and reliability of code.
\vglcite[20,22]{tiwariCOMPONENTBASEDSOFTWAREENGINEERING2024}

This is worth noting, since it is a core principle of modern web development and UI Design.
After the release of Angular in 2010, this concept really started gaining traction in web
development. \vglcite{angularjsAngularJS}
% NOTE: Maybe say that in 2013 Atomic Design was introduced by Brad Frost and got UI Design on board
Now, almost all popular frontend frameworks like React, Vue or Svelte are built around components.
\vglcite{reactReact} \vglcite{vuejsVueJS} \vglcite{svelteSvelte}

Around the same time as Component Based Design became popular in web development, Brad Frost
introduced Atomic Design, which translates this programming concept to UI Design. In section
\ref{Atomic Design Systems}, the concept of Atomic Design and how it can be used practically will be
explored in more detail. Essentially Frost took chemistry as an inspiration and composed a methodology
using five stages called Atoms, Molecules, Organisms, Templates and Pages that should serve as a
\textit{mental model} for UI Designers. \vglcite[42]{frostAtomicDesign2016}

Today, Component Based UI Design is a deep rooted part of almost every Interface Design tool as
well. That is why it is incredibly crucial for designers to understand this concept and how it can
be effectively applied to their work. Since, in theory, this approach brings design and development
closer to the ideal of seamless collaboration, resulting in interfaces that are both beautiful,
functional, and maintainable.