\newpage
\subsection{Component Based UI Design}
UI Designs without actually coding them are like having raw cookie dough pieces without an oven.
They are unfinished and can't be properly consumed. And an oven alone will not make the cookie dough
itself. It is the combination of the two that makes delicious cookies. UI Design depends on
development and vice versa.

Because of this strong dependency, it only makes sense that both disciplines share common concepts
and principles. One of these concepts is called Component Based Design.

\subsubsection{The Concept of Components}
This concept originates from a programming paradigm called Component Based Software Engineering
(CBSE). Without getting too much into the technical details, CBSE is about breaking down software
into smaller, independent and reusable parts called components. This modularity allows for easier
maintenance, scalability and reliability of code.
\vglcite[20,22]{tiwariCOMPONENTBASEDSOFTWAREENGINEERING2024}

I mention this specifically, since it is a core principle of modern web development and UI Design.
After the release of Angular in 2010, this concept started really gaining traction in web development.
(https://angularjs.org/)
% NOTE: Maybe say that in 2013 Atomic Design was introduced by Brad Frost and got UI Design on board
Now, almost all popular frontend frameworks like React, Vue or Svelte are built around components.
(https://react.dev/) (https://vuejs.org/guide/introduction.html) (https://svelte.dev/)

Around the same time as Component Based Design became popular in web development, Brad Frost
introduced Atomic Design, which translates this programming concept to UI Design. I will go into
more detail about what Atomic Design is and how to practially use it in later chapters. Essentially
he took chemistry as an inspiration and composed a methology using five stages called Atoms,
Molecules, Organisms, Templates and Pages that should serve as a \textit{mental model} for UI
Designers. \vglcite[42]{frostAtomicDesign2016}

Today, Component Based UI Design is a deep rooted part of almost every Interface Design tool as
well. That is why I believe it is incredibly crucial for designers to understand this concept and
how it can be applied to their work. Since, in theory, it would allow them to create designs that go
hand in hand with the development process. And isn't that the perfect picture we are all striving
for? Design and development, skipping hand in hand through the meadow, leaving behind a trail of
beautiful, functional and maintainable interfaces.

\begin{figure}[]
    \begin{center}
        \includegraphics[width=200pt, height=\textheight,keepaspectratio]{2_chapter/designDevHandInHand.jpg}
        \caption[Caption (url)]{Design and development, skipping hand in hand through the meadow}
    \end{center}
\end{figure}

