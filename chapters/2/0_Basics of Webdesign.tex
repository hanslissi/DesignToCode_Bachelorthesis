\section{Basics of Webdesign}
Optimizing the workflow from User Interface Design to code implementation requires to have a solid
foundational understanding of what UI Design and development is. More specifically, understanding
web-specific aspects of these fields is crucial to optimize the workflow for web projects. In this
chapter, we will discuss the basics of User Interface Design,

\subsection{User Interface Design}
The interaction design foundation defines User Interface (UI) Design as "the process designers use
to build interfaces in software or computerized devices, focusing on looks or style. Designers aim
to create interfaces which users find easy to use and pleasurable. UI design refers to graphical
user interfaces and other forms—e.g., voice-controlled interfaces."
\directcite{interactiondesignfoundation-ixdfWhatUserInterface2016}.

I believe this definition is a good starting point to understand what UI design is about, because it
shows that it's about more than just the visual. User interfaces are everywhere, be it the ATM, the
stove, the bike computer or the time tracking app in the browser. Each one of them needs to be
carefully designed. It's about finding ways to make this human-computer interaction as smooth,
efficient and delightful as possible.
\vglcite{interactiondesignfoundation-ixdfWhatUserInterface2016}



\subsubsection{User Experience and Usability}
To grasp the concept of User Experience, or UX for short, we need to zoom out of the context of User
Interfaces. UX is about looking at all kinds of touchpoints a user has with a product or service and
seeing the holistic experience. This of course includes the UI, but also other domains such as
content quality, customer support, branding, and more. \vglcite{normanDefinitionUserExperience1998}

In other words, even an app like Spotify with a great UI, would have a poor UX if there were only 10
songs available.

Usability on the other hand is, as Jacob Nielsen suggests, a quality attribute of Interfaces. This
attribute indicates how easy it is for users to use a product, by observing and measuring the five
quality components: learnability, efficiency, memorability, errors, and satisfaction
% NOTE: Do I need to explain these further?
defined by Nielsen. \vglcite{nielsenUsability101Introduction2012}

\subsubsection{Web Design}

\subsubsection{Tools}
There are many great tools out there that aim to make UI Design easier and more efficient. Some of
them also have features that help to bridge the gap between design and code. I selected the
mentioned tools based on their popularity, since I aim to provide a solution that tailors to the
majority of designers and developers.

(https://uxdesign.cc/figma-continues-to-skyrocket-63-reported-it-was-their-primary-ui-design-tool-in-2021-bb9390a8b96b)

Figma: Figma is by far the most popular UI Design tool at the moment. It's a cloud-based design tool
that focuses on real-time collaboration, prototyping and design system creation. The majority of
features is free to use, which makes it very approachable. Apart from their frequent updates and
improvements, Figma has a lively community allowing for the creation of plugins, templates and other
resources. Having robust design system features like components, style definitions and variables,
Figma is one of the leading tools in the UI Design space.
(https://help.figma.com/hc/en-us/categories/360002042553)

Sketch: Sketch is a design tool that was first released in 2010. Although, Figma has surpassed
Sketch in popularity, it still has a large user base. It offers similar features to Figma like
real-time collaboration, components or prototyping. However, aside from the free, browser-accessible
file viewer, Sketch is a paid tool available only on macOS. The program has a strong focus on
creating design systems and has a large library of plugins and templates. (https://www.sketch.com/)

Adobe Illustrator and Adobe Photoshop: Although the vector graphics editor Adobe Illustrator and the
raster graphics editor Adobe Photoshop are not specifically tailored to UI Design, they are widely
used by designers. Many designers report that they use them as secondary tools aiding in the
creation of more complex graphics. (https://uxtools.co/survey/2023/ui-design/)

In future chapters, I will generally refer to Figma, since it is the most popular tool at the moment
and also the one I am most familiar with. Although the majority of concepts and techniques mentioned
will be applicable to other tools as well.

\subsection{Component Based UI Design}
UI Designs without actually coding them are like having raw cookie dough pieces without an oven.
They are unfinished and can't be properly consumed. And an oven alone will not make the cookie dough
itself. It is the combination of the two that makes delicious cookies. UI Design depends on
development and vice versa.

Because of this strong dependency, it only makes sense that both disciplines share common concepts
and principles. One of these concepts is called Component Based Design.

\subsubsection{The Concept of Components}
This concept originates from a programming paradigm called Component Based Software Engineering
(CBSE). Without getting too much into the technical details, CBSE is about breaking down software
into smaller, independent and reusable parts called components. This modularity allows for easier
maintenance, scalability and reliability of code.
\vglcite[20,22]{tiwariCOMPONENTBASEDSOFTWAREENGINEERING2024}

I mention this specifically, since it is a core principle of modern web development and UI Design.
After the release of Angular in 2010, this concept started gaining traction in web development.
(https://angularjs.org/)
% NOTE: Maybe say that in 2013 Atomic Design was introduced by Brad Frost and got UI Design on board
Now, almost all popular frontend frameworks like React, Vue or Svelte are built around components.
(https://react.dev/) (https://vuejs.org/guide/introduction.html) (https://svelte.dev/)

Around the same time as Component Based Design became popular in web development, Brad Frost
introduced Atomic Design, which translates this programming concept to UI Design. I will go into
more detail about what Atomic Design is and how to practially use it in later chapters. Essentially
he took chemistry as an inspiration and composed a methology using five stages called Atoms,
Molecules, Organisms, Templates and Pages that should serve as a \textit{mental model} for UI
Designers. \vglcite[42]{frostAtomicDesign2016}

Today, Component Based UI Design is a deep rooted part of almost every Interface Design tool as
well. That is why I believe it is incredibly crucial for designers to understand this concept and
how it can be applied to their work. Since, in theory, it would allow them to create designs that go
hand in hand with the development process. And isn't that the perfect picture we are all striving
for? Design and development, skipping hand in hand through the meadow, leaving behind a trail of
beautiful, functional and maintainable interfaces.

\begin{figure}[h!]
    \begin{center}
        \includegraphics[width=200pt, height=\textheight,keepaspectratio]{2_chapter/designDevHandInHand.jpg}
        \caption[Caption (url)]{Design and development, skipping hand in hand through the meadow}
    \end{center}
\end{figure}

\subsubsection{Atomic Design and Recent Developments}
So, let us take a closer look at Atomic Design. \\\\
\textbf{Atoms} \\
Atoms are the smallest building blocks of the design. They are the basic elements like buttons,
textfields or icons. Things that cannot be broken down any further. \\\\
\textbf{Molecules} \\
Molecules are groups of atoms that together form a simple group of UI. For example, combining
a headline, an icon and filter buttons it would form a neat filter bar. \\\\
\textbf{Organisms} \\
Organisms are more complex groups of UI elements. They can be a combination of atoms, molecules and
also other organisms. When combining the filter bar molecule, a few buttons and a blog post grid
organism, it would form a functional blog post overview organism. \\\\
\textbf{Templates} \\
Templates are like blueprints of pages. They consist of organisms and molecules and define the
layout of a page, giving actual meaning to the UI elements. \\\\
\textbf{Pages}\\
Pages take templates as a basis and fill them with actual content. These represent how the final
delivered product will or should look like. \\
\vglcite[43,53]{frostAtomicDesign2016}

Creating hierarchical systems like this has many benefits.

\textbf{Change it once, change it everywhere}\\

\textbf{Consistency}\\

\textbf{Adhere to the Single Responsibility Principle}\\

\textbf{Support iterative workflows}\\

And many more.


\subsection{Design Systems}
\subsubsection{Parts of the Design System}
% Just explain the parts: 
% - Style Guide
% - Pattern Library
% - Component Library
% - Design Tokens
\subsubsection{Common Best Practices for Design Systems}