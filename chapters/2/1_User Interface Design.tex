\subsection{User Interface Design}
The interaction design foundation defines User Interface (UI) Design as "the process designers use
to build interfaces in software or computerized devices, focusing on looks or style. Designers aim
to create interfaces which users find easy to use and pleasurable. UI design refers to graphical
user interfaces and other forms—e.g., voice-controlled interfaces."
\directcite{interactiondesignfoundation-ixdfWhatUserInterface2016}.

I believe this definition is a good starting point to understand what UI design is about, because it
shows that it's about more than just the visual. User interfaces are everywhere, be it the ATM, the
stove, the bike computer or the time tracking app in the browser. Each one of them needs to be
carefully designed. It's about finding ways to make this human-computer interaction as smooth,
efficient and delightful as possible.
\vglcite{interactiondesignfoundation-ixdfWhatUserInterface2016}

% =================================================================================================

\subsubsection{User Experience and Usability}
To grasp the concept of User Experience, or UX for short, we need to zoom out of the context of User
Interfaces. UX is about looking at all kinds of touchpoints a user has with a product or service and
seeing the holistic experience. This of course includes the UI, but also other domains such as
content quality, customer support, branding, and more. \vglcite{normanDefinitionUserExperience1998}

In other words, even an app like Spotify with a great UI, would have a poor UX if there were only 10
songs available.

Usability on the other hand is, as Jacob Nielsen suggests, a quality attribute of Interfaces. This
attribute indicates how easy it is for users to use a product, by observing and measuring the five
quality components: learnability, efficiency, memorability, errors, and satisfaction
% NOTE: Do I need to explain these further?
defined by Nielsen. \vglcite{nielsenUsability101Introduction2012}

\subsubsection{Web Design}

\subsubsection{Tools}
There are many great tools out there that aim to make UI Design easier and more efficient. Some of
them also have features that help to bridge the gap between design and code. I selected the
mentioned tools based on their popularity, since I aim to provide a solution that tailors to the
majority of designers and developers.

(https://uxdesign.cc/figma-continues-to-skyrocket-63-reported-it-was-their-primary-ui-design-tool-in-2021-bb9390a8b96b)

Figma: Figma is by far the most popular UI Design tool at the moment. It's a cloud-based design tool
that focuses on real-time collaboration, prototyping and design system creation. The majority of
features is free to use, which makes it very approachable. Apart from their frequent updates and
improvements, Figma has a lively community allowing for the creation of plugins, templates and other
resources. Having robust design system features like components, style definitions and variables,
Figma is one of the leading tools in the UI Design space.
(https://help.figma.com/hc/en-us/categories/360002042553)

Sketch: Sketch is a design tool that was first released in 2010. Although, Figma has surpassed
Sketch in popularity, it still has a large user base. It offers similar features to Figma like
real-time collaboration, components or prototyping. However, aside from the free, browser-accessible
file viewer, Sketch is a paid tool available only on macOS. The program has a strong focus on
creating design systems and has a large library of plugins and templates. (https://www.sketch.com/)

Adobe Illustrator and Adobe Photoshop: Although the vector graphics editor Adobe Illustrator and the
raster graphics editor Adobe Photoshop are not specifically tailored to UI Design, they are widely
used by designers. Many designers report that they use them as secondary tools aiding in the
creation of more complex graphics. (https://uxtools.co/survey/2023/ui-design/)

In future chapters, I will generally refer to Figma, since it is the most popular tool at the moment
and also the one I am most familiar with. Although the majority of concepts and techniques mentioned
will be applicable to other tools as well.
