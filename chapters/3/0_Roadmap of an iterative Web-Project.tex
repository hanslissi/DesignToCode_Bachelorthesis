\newpage
\section{Roadmap of an iterative Web-Project} % NOTE Maybe change title to include Project Managemnt
% Say that for better collaboration not only handoff but also project management is crucial.
% Why is traditional project management not really working. Agile is good (adoption grows and is now
% the new standard, find statistics), lean ux needs to be integrated, so also design can be as
% iterative as (possible) development.
In order to achieve the best collaboration between designers and developers, not only handoff
quality is important, but also the way project management is handled. It shapes how the team sees
the project and how they work together. In this chapter, I will outline how web projects are usually
set up, what challenges they face at different stages, the potential consequences of these
challenges and possible solutions for overcoming them.

\subsection{Traditional Project Management}
% Explain waterfall model (maybe a little sarcastic painting the perfect world) and then oops yeah
% that's why it is not really working for web projects. Cause working in Silos is bad.
% First quickly explain waterfall model.
% Then say advantages and disadvantages.
% Then say why it is not working for web projects.
%  - After project finish there is a product, but in between there is almost never a working product.
%  - Silos are bad, long handoffs, no way to change design after development started.
%  - No way to react to user feedback.
%  - No way to react to changing requirements.
The most traditional way of managing software projects is the Waterfall model. It breaks down the
project into several phases, where each phase is completed before the next one starts. That means
it's a very linear approach, the project \textit{flows} from one phase to the next, with handoffs in
between. \vglcite{theinstituteofprojectmanagementWaterfallMethodology2022}

For a typical software or web project, the phases are usually something like this: % NOTE: Maybe add a graphic
\begin{enumerate}
    \item Requirements
    \begin{description}
        \item Conduct research and gather requirements for the project. Collect as much information
        as possible to then create a detailed project plan.
    \end{description}
    \item Design
    \begin{description}
        \item Create the design based on the requirements that show how the final product will look
        and how this will be achieved. This includes UI, but also architecture design. 
    \end{description}
    \item Implementation \& Testing
    \begin{description}
        \item Develop the product based on the design and test it to ensure it meets the
        requirements.
    \end{description}
    \item Verification \& Integration
    \begin{description}
        \item The product is validated to check if all requirements are met. If everything works,
        the product is launched.
    \end{description}
    \item Maintenance
    \begin{description}
        \item After launch the product is maintained and updated as needed.
    \end{description}
\end{enumerate}

When looking at the sequence of these phases, it may seem like it's a perfect solution for all
projects. And although, this model works especially for projects that have a defined and fixed
scope, many software or web projects don't exactly work that way. Also, teams of different
disciplines work separately from each other. \vglcite{theinstituteofprojectmanagementWaterfallMethodology2022} 

So, let's look at some common problems and their effects, to understand what can be optimized in the
workflow.

As UX/UI Designers, we want to create the best product for the target audience. This means that the
design process is not a one-time thing. It's an iterative process that 

\subsubsection{Common Problems and Effects of Bad Collaboration}
\textbf{Long Handoffs} \\
When the design is finished and handed off to the development team, you can imagine that it's going
to be a long meeting. The developers need to understand the design from the ground up, which can
lead to misunderstandings, misinterpretations and a lot of back and forth.\\\\
\textbf{No end-user prioritization} \\
The focus on meeting requirements first specified, can lead to a product that doesn't actually comply
with the needs of the user. As there is almost no way to go back and change the design, the products
success is only really revealed after launch which is very risky.\\\\
\textbf{No UX focus} \\
Adding to the previous point, without much end-user feedback usability testing and other UX methods are
cut short making it hard to emphazise UX. Even if usability testing is integrated, it is extremely
costly to go back and change the initial design in later phases such as implementation.\\\\
\textbf{Risk of nothing to show for} \\
Another result of the linear, separate team approach is that there isn't a working version of the
product for quite some time. So, when the project is cancelled prematurely, there could be nothing
to show for.\\
\vglcite{theinstituteofprojectmanagementWaterfallMethodology2022}
% NOTE: mach so ne grafik rein:
% https://www.google.com/search?sca_esv=61eddb219bf0e5e6&sxsrf=AHTn8zo4RD4tpO9VjZHRDRIlptCIvp0pGQ:1741695083635&q=project+change+cost+over+project+timeline+graph%C3%A4&udm=2&fbs=ABzOT_CWdhQLP1FcmU5B0fn3xuWpA-dk4wpBWOGsoR7DG5zJBkzPWUS0OtApxR2914vrjk4ZqZZ4I2IkJifuoUeV0iQtITiOPPo9tDzmt9ZPGYJiIba3ipclDVbOjJlvTbgEP2s-bkOIhr5ELgbQI8I7zKhriYCgRXaYljMf-YpaNgLRzy2fJ38VbFwBTF_D5ZCA5_SutZQD&sa=X&ved=2ahUKEwiplpHm_4GMAxW6S_EDHckeGToQtKgLegQIFxAB&biw=1680&bih=913&dpr=2#vhid=jRm_exLG44TCIM&vssid=mosaic 
\subsection{Agile Project Management \& Lean UX}
% Explain what Agile is, why it is good, how it is adopted and why it is the new standard =>
% statistics. Mention that agile with a focus on design is not that easy and that there are mulitple
% popular ways. 
Now that we understand the problems of traditional project management, let's see how Agile and Lean
UX can help to optimize the workflow.

\subsubsection{Agile Manifesto}

What being Agile means is defined in the Agile Manifesto. Created in 2001, it is a set of values and
principles, written by 17 software developers, that aim to improve the way software is made. The
four core values are:
\begin{enumerate}
    \item Individuals and interactions over processes and tools
    \item Working software over comprehensive documentation
    \item Customer collaboration over contract negotiation
    \item Responding to change over following a plan
\end{enumerate}
\directcite{beckManifestoAgileSoftware2001}

So, only by looking at these values, it immediately becomes clear how different this approach is.
Instead of having separate teams, Agile promotes collaboration between all members. Instead of a 
fixed scope and plan, Agile is about being able to react to change and feedback. 

\subsubsection{Agile Frameworks}
To actually implement Agile, there are different frameworks that can be used. The most popular one
is Scrum. It tries to realize the Agile values by breaking down the project into smaller, manageable
parts called sprints. Each sprint is about 2-4 weeks long and aims to \textit{increment} the product
a little bit each time. \vglcite[7]{schwaberScrumGuideDefinitive2020}

The basic idea is that after each sprint, the team has a working version of the product that can be
tested and improved the next sprint. Of course in the beginning the product is basic, but it
iteratively grows more mature. 

In contrast to the Waterfall model, where the scope is fixed, Agile has a fixed time and budget.
This illustration shows the difference between the two models:
% NOTE:  Mention the triangle:
% https://www.ecosia.org/images?addon=firefox&addonversion=5.1.1&q=agile+vs+traditional+triangle#id=88D37B800865C1E5C05FA187E99967170A4BCFEF 

Scrum also focuses on meaningful meetings, like the daily standup, where the whole developer team
comes together discussing what each one did the previous day, what they will do today and if there
are any struggles. So everyone is involved and can help where help is needed.
\vglcite[9]{schwaberScrumGuideDefinitive2020} 

% NOTE: Add graphics of scrum process

That brings us a lot closer to what we want to achieve for the design of web projects. Iteratively
building the solution allows us being able to react to feedback and changes and always having a
fitting, working product. However, where does design fit in all of this? That's where Lean UX comes
in.

\subsubsection{Lean UX}
% For example Dual Track Agile. Then explain why Lean UX has the best things for my
% ways since small teams work well with it to be integrated, so also design can be as iterative as
% development. Especially working well in small teams. Maybe mention Dual-Track Agile and also the
% risk of it creating silos again.

% Hmm lets skip dual track agile and not mention that much about it. It is however also mentioned in
% Lean UX so lets see how I can integrate that.
In their book \textit{Lean UX}, Jeff Gothelf and Josh Seiden describe how Lean UX is a way of
integrating UX/UI Design into Agile software projects. They build on the core values of the Agile
Manifesto and prove that design can be as iterative as development.  

The main idea is to work in small, cross-functional teams that build a shared understanding of the
product, the customers and the business goals. These teams of different disciplines continously work
closely together, which minimizes the need for long handoffs and has the potential to create better
solutions, since more perspectives are included. 
\vglcite[24, 26]{gothelfLeanUXProduktentwicklung2016} 

So, how does this process actually look like? The process is broken down into four main parts:

% NOTE: Maybe add a graphic of the Lean UX process

\textbf{Outcomes, Assumptions, Hypotheses} \\ % (Lean UX page 38)
Designers and non-designers get together to formulate assumptions, guided by high quality problem
statements. These assumptions contain statements about what the team believes might be true for the
product given the constraints of the problem statement. Then assumptions are turned into hypotheses
that can be designed and tested by their expected outcomes. After prioritizing the hypotheses by
looking at the value and risk, the team can start designing.
\vglcite[38, 62]{gothelfLeanUXProduktentwicklung2016}\\\\
\textbf{Design it} \\
Designers call informal collaborative design meetings or so-called Design Stuidos, where the
cross-functional team does brainstorming, sketching or one-to-one sessions. In these sessions
everyone gets to come up with ideas and sketches of low-fidelity. That way the designers have a
great pool of diverse ideas to work with. \vglcite[63, 67]{gothelfLeanUXProduktentwicklung2016} \\\\ 
\textbf{Create an MVP} \\
Creating an Minimal Viable Product (MVP) has the goal of learning as much about the target audience
as possbile in the least amount of time. With the hypotheses and the designs in hand an MVP can be
created by asking a question like \textit{What is the least amount of work we can do to achieve
this?}. By doing this, no resources are wasted to do something which brings no or little value and
the team can implement features quickly and informed.
\vglcite[92, 93]{gothelfLeanUXProduktentwicklung2016} \\\\
\textbf{Research \& Learning} \\
With the MVP in place, testing can begin to validate assumptions and hypotheses. With collaborative
research techniques or customer feedback, again every team member gets in contact with this aspect
of building a product adding to the shared understanding of it. These learnings feed the next cycle
of this process. \vglcite[110, 113]{gothelfLeanUXProduktentwicklung2016} \\

As you can see in between this process there are also several fun and innovative methods that push
collaboration even further. Like the Design Studio, where the whole team comes together to sketch
out ideas for specific problems which is led by a designer, but everyone contributes. 
\vglcite[68,69]{gothelfLeanUXProduktentwicklung2016} Or collaborative design, where 2 or more
members come together for a specific problem and sketch ideas rather informally. 
\vglcite[66]{gothelfLeanUXProduktentwicklung2016}

These practices all lead to more shared knowledge and understanding and help developers be part of
the design process and vice versa. The whole team moves from blindly implementing features to seeing
the holistic picture and focusing on solving real-world problems - the team culture is positively
different.

As suggested by the authors, this process can also be integrated with the Scrum framework. I believe
makes a lot of sense, since Scrum is by far the most used Agile framework.
\vglcite[13]{kpmgAgileTransformationAgile2019} 
% NOTE: is there more to say about this? Like how do they suggest it to be integrated? (Lean UX page
% 140)

\subsection{Conclusion}
% To make great User Interfaces and User Experiences, it is not enough for designers to do their job
% and developers implementing designs. All disciplines need to work towards that goal and understand
% why something is done and why it is that way. Everyone should be able to understand why the stop
% button of their timer app is that big and at the top right, instead of mini and at the bottom
% left.

As seen in this chapter, the way a project is managed and what \textit{rules} the team follows, have 
a huge impact on the outcome. While the traditional Waterfall model can work for projects with a
fixed scope, it may not be the way to go for design focused web projects as they are usually prone
to change while in development.

To make great User Interfaces and User Experiences, it is crucial that all disciplines work closely
together and understand the \textit{why} behind decisions. Agile methods like Lean UX and the
combination with Scrum seem to be one of the best ways to achieve this change in culture.
Breaking down silos and integrating engineers into the design process will lead to better, more
technically feasible, and developer friendly designs. 

Promoting more informal meetings like collaborative design sessions can also help designers and
developers to understand requirements of each others work better. 
