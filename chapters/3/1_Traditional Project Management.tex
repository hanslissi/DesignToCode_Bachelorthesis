\newpage
\subsection{Traditional Project Management}
% Explain waterfall model (maybe a little sarcastic painting the perfect world) and then oops yeah
% that's why it is not really working for web projects. Cause working in Silos is bad.
% First quickly explain waterfall model.
% Then say advantages and disadvantages.
% Then say why it is not working for web projects.
%  - After project finish there is a product, but in between there is almost never a working product.
%  - Silos are bad, long handoffs, no way to change design after development started.
%  - No way to react to user feedback.
%  - No way to react to changing requirements.
The most traditional way of managing software projects is the Waterfall model. It breaks down the
project into several phases, where each phase is completed before the next one starts. That means
it is a very linear approach, the project \textit{flows} from one phase to the next, with handoffs in
between. \vglcite{theinstituteofprojectmanagementWaterfallMethodology2022}

For a typical software or web project, the phases are usually something like this: % NOTE: Maybe add a graphic
\begin{enumerate}
    \item Requirements
    \begin{description}
        \item Conduct research and gather requirements for the project. Collect as much information
        as possible to then create a detailed project plan.
    \end{description}
    \item Design
    \begin{description}
        \item Create the design based on the requirements that show how the final product will look
        and how this will be achieved. This includes UI, but also architecture design. 
    \end{description}
    \item Implementation \& Testing
    \begin{description}
        \item Develop the product based on the design and test it to ensure it meets the
        requirements.
    \end{description}
    \item Verification \& Integration
    \begin{description}
        \item The product is validated to check if all requirements are met. If everything works,
        the product is launched.
    \end{description}
    \item Maintenance
    \begin{description}
        \item After launch the product is maintained and updated as needed.
    \end{description}
\end{enumerate}

When looking at the sequence of these phases, it may seem like it is a perfect solution for all kinds
of projects. However, while this model works especially for projects that have a defined and fixed
scope, many software or web projects don't exactly work that way. Also, teams of different
disciplines work separately from each other, forming silos.
\vglcite{theinstituteofprojectmanagementWaterfallMethodology2022}  

So, let's look at some common problems and their effects, to understand what can be optimized in the
workflow.

As UX/UI Designers, we want to create the best product for the target audience. This means that the
design process is not a one-time thing. It is an iterative process that 

\subsubsection{Common Problems and Effects of Bad Collaboration}
\textbf{Long Handoffs} \\
When the design is finished and handed off to the development team, you can imagine that it is going
to be a long meeting. The developers need to understand the design from the ground up, which can
lead to misunderstandings, misinterpretations and a lot of back and forth.\\\\
\textbf{No end-user prioritization} \\
The focus on meeting requirements first specified, can lead to a product that doesn't actually comply
with the needs of the user. As there is almost no way to go back and change the design, the products
success is only really revealed after launch which is very risky.\\\\
\textbf{No UX focus} \\
Adding to the previous point, without much end-user feedback usability testing and other UX methods are
cut short making it hard to emphazise UX. Even if usability testing is integrated, it is extremely
costly to go back and change the initial design in later phases such as implementation.\\\\
\textbf{Risk of nothing to show for} \\
Another result of the linear, separate team approach is that there isn't a working version of the
product for quite some time. So, when the project is cancelled prematurely, there could be nothing
to show for.\\
\vglcite{theinstituteofprojectmanagementWaterfallMethodology2022}
% NOTE: mach so ne grafik rein:
% https://www.google.com/search?sca_esv=61eddb219bf0e5e6&sxsrf=AHTn8zo4RD4tpO9VjZHRDRIlptCIvp0pGQ:1741695083635&q=project+change+cost+over+project+timeline+graph%C3%A4&udm=2&fbs=ABzOT_CWdhQLP1FcmU5B0fn3xuWpA-dk4wpBWOGsoR7DG5zJBkzPWUS0OtApxR2914vrjk4ZqZZ4I2IkJifuoUeV0iQtITiOPPo9tDzmt9ZPGYJiIba3ipclDVbOjJlvTbgEP2s-bkOIhr5ELgbQI8I7zKhriYCgRXaYljMf-YpaNgLRzy2fJ38VbFwBTF_D5ZCA5_SutZQD&sa=X&ved=2ahUKEwiplpHm_4GMAxW6S_EDHckeGToQtKgLegQIFxAB&biw=1680&bih=913&dpr=2#vhid=jRm_exLG44TCIM&vssid=mosaic 
