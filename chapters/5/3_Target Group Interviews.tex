\newpage
\subsection{Target Group Interviews}
In order to develop a solution that is truly valuable, it is important to emphasize with the target
group. Directly talking to the people that are part of the process in their day-to-day business
allows exactly for that.

\subsubsection{Goal of the Interviews}
There were two main goals. Firstly, finding more real world experiences and uncovering problems yet
to be found. Secondly, getting immediate feedback on the practical work, so that a quick iteration
can take place. It was a conscious decision to conduct the interviews after having worked on the
solution for some time, since there was already a tangible product to be reviewed by the designer
and developer respectively.

\subsubsection{Participants}
Mary -  UI/UX Student Trainee - 1 year experience
Peter - Frontend Developer - 10+ years experience

%TODO: Quick "Persona" view of those saying changed the names

\subsubsection{Presentation and Conclusion of the Results}
\textbf{Designer Interview}
Mary states that the developers she worked with until now had little to no knowledge of design tools
like Figma. (vgl. line 74) "One of them was interested, but had only just learnt how to work with
it." (line 89) She thinks this lack of knowledge on developer side is a serious issue and wishes
"that all [developers] know Figma" (line 542)

Handing off designs is done by manually exporting finished UI components as SVG files and
uploading them to Confluence, a platform developers are comfortable with. She states that "you lose
time if you export everything to Confluence and always have to write a description" (line 680). Even 
though documentation is important this extra step seems unnecessary.

Mary has also experienced major discrepancies between design and implementation in one of her
earliest projects. The finished product "looked totally different" (line 452), even buttons had
completely different colors. According to her, developers not knowing Figma, budget-cuts and
unqualified engineers where the primary reasons for this (line 476 - 503)

\textbf{Developer Interview}
In contrast to the developers Mary worked with, Peter has worked with Figma for quite some time.
Together with their external UI designer, they built a solid design system for their product. He
notes that while a thought-out structure is not immediately visible and important, it "helps if
there is a certain system over everything, because then you also understand in the code what the
system is and can then map the system in Code". (line 170-173) They also "use the same design tokens 
in Figma and in the code". (line 146)

His team often notices that designers make decisions, "which [are] in principle a good idea, but
[the designer] doesn't know what impact this will have on the implementation". (line 281-284) As he
and his team try to minimize unnecessary overload, they evaluate and question designs if they notice
high implementation complexity. They check with the designer if there isn't an easier solution with
a similar value. (vgl 290-338)

Mentioned practices that speed up development include easily findable and exportable assets (vgl
416-434), unified breakpoints used in both Figma and Code (vgl 863-866) and responsive components in
desktop, tablet and mobile versions as variants next to each other in Figma (vgl 554-566).

\textbf{Conclusion}
The insights of the interviews done allow for further fine tuning and adapting the content of the
practical work. In the next step this tuning process is further elaborated with specific examples.
