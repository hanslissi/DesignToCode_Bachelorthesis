\newpage
\subsection{Target Group Interviews}
In order to develop a solution that is truly valuable, it is important to emphasize with the target
group. Directly talking to the people that are part of the process in their day-to-day business
allows exactly for that.

\subsubsection{Goal of the Interviews}
There were two main goals. Firstly, finding more real world experiences and uncovering problems yet
to be found. Secondly, getting immediate feedback on the practical work, so that a quick iteration
can take place. It was a conscious decision to conduct the interviews after having worked on the
solution for some time, since there was already a tangible product to be reviewed by the designer
and developer respectively.

\subsubsection{Participants}
The names as well as the company the participants work for have been anonymized, as this data is not
necessary for the analysis. 
\begin{figure}[H]
    \centering
    \includegraphics[width=300pt]{Chapter 5/Personas.png}
    \caption{Interview participants (Source: own illustration)}
\end{figure}

%TODO: Quick "Persona" view of those saying changed the names

\subsubsection{Presentation and Conclusion of the Results}
\textbf{Designer Interview}
Interviewee A states that the developers she worked with until now had little to no knowledge of
design tools like Figma. \vglcite[74]{interviewpartnerinaInterview1Gefuehrt2025a} "One of them was
interested, but had only just learnt how to work with it."
\directcite[89]{interviewpartnerinaInterview1Gefuehrt2025a} She thinks this lack of knowledge on
developer side is a serious issue and wishes "that all [developers] know Figma".
\directcite[542]{interviewpartnerinaInterview1Gefuehrt2025a}

Handing off designs is done by manually exporting finished UI components as SVG files and
uploading them to Confluence, a platform developers are comfortable with. She states that "you lose
time if you export everything to Confluence and always have to write a description".
\directcite[680]{interviewpartnerinaInterview1Gefuehrt2025a} Even though documentation is important
this extra step seems unnecessary to her.

Interviewee A has also experienced major discrepancies between design and implementation in one of her
earliest projects. The finished product "looked totally different"
\directcite[452]{interviewpartnerinaInterview1Gefuehrt2025a}, even buttons had completely different
colours. According to her, developers not knowing Figma, budget-cuts and unqualified engineers where
the primary reasons for this \directcite[476, 503]{interviewpartnerinaInterview1Gefuehrt2025a}

\textbf{Developer Interview}
In contrast to the developers interviewee A worked with, interviewee B has worked with Figma for
quite some time. Together with their external UI designer, they built a solid design system for
their product. He notes that while a thought-out structure is not immediately visible and important,
it "helps if there is a certain system over everything, because then you also understand in the code
what the system is and can then map the system in Code".
\directcite[170,173]{interviewpartnerbInterview2Gefuehrt2025a} They also "use the same
design tokens in Figma and in the code". \directcite[146]{interviewpartnerbInterview2Gefuehrt2025a}

His team often notices that designers make decisions, "which [are] in principle a good idea, but
    [the designer] doesn't know what impact this will have on the implementation".
\directcite[281,284]{interviewpartnerbInterview2Gefuehrt2025a} As he and his team try to minimize
unnecessary overload, they evaluate and question designs if they notice high implementation
complexity. They check with the designer if there is not an easier solution with
a similar value. \vglcite[290,338]{interviewpartnerbInterview2Gefuehrt2025a}

Mentioned practices that speed up development include
\begin{itemize}
    \item easily findable and exportable assets \vglcite[416,434]{interviewpartnerbInterview2Gefuehrt2025a},
    \item unified breakpoints used in both Figma and Code
          \vglcite[863,866]{interviewpartnerbInterview2Gefuehrt2025a} and
    \item responsive components in desktop, tablet and mobile versions as variants next to each
          other in Figma. \vglcite[554-566]{interviewpartnerbInterview2Gefuehrt2025a}
\end{itemize}

\textbf{Conclusion}
The insights of the interviews done allow for further fine tuning and adapting the content of the
practical work. In the next step this tuning process is further elaborated with specific examples.
