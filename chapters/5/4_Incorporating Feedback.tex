\newpage
\subsection{Incorporating Feedback}
This chapter explains the process of incorporating the feedback from the target group interviews
into the current progress of The DesignAPI. Since many changes have been made, only the most impactful
have been highlighted here.

As interviewee A mentioned that numerous engineers have never been in contact with Figma, a new
category was added to fill this gap.

\textbf{Figmafication} - Focuses on getting developers familiar with using Figma. The
opportunity for this category is to increase the number of engineers actively
incorporating design tools like Figma into their workflows.

Having unified screen breakpoints and keeping responsive components variants close to each other was
mentioned by interviewee B as a very helpful practice. This resulted in the following two cards
being added to the collection.
%TODO DONE: Add graphics of the two responsiveness cards 

Lastly, interviewee Bs example of a situation where a simple Radio-Button would have done the job
perfectly, but instead the designer came up with a much more complex solution was a powerful story.
Identifying parts of the design that would cost much time to implement, but are valuable in contrast
to parts that do not provide enough value and communicating this early to the designer is important.
\vglcite[281, 389]{interviewpartnerbInterview2Gefuehrt2025a}
It enables the team to deliver value at parts where it actually matters. This sentiment has been
turned into a card as well as a message template.
%TODO DONE: Add graphics of card and message template.

\begin{figure}[H]
    \centering
    \includegraphics[width=400pt]{Chapter 5/New Cards.png}
    \caption{New cards in the Responsiveness category based on feedback (Source: own illustration)}
\end{figure}

\begin{figure}[H]
    \centering
    \includegraphics[width=400pt]{Chapter 5/New Card and template.png}
    \caption{New card and message template in the Specials category based on feedback (Source: own illustration)}
\end{figure}
