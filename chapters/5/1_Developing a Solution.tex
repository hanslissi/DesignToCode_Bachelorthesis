\newpage
\subsection{Developing a Solution}
Creating a solution that would tailor to the struggles of both designers and engineers 

\subsubsection{The Vision of High Quality Collaboration}
Before starting to think about solutions, the actual needs of the parties have to be examined and a
clear vision needs to be set. To do this, the data of the survey done beforehand was used as a
foundation. 

\begin{description}
    \item[The need for cross-disciplinary knowledge] - Both designers and developers would benefit
    from more knowledge about the other discipline at places where it matters most. If designers
    knew why using variables for design tokens or Figma Auto-Layout in frames matters to devs, they
    could use these things more intentionally. Similarly, if developers knew how to use design tools
    to their advantage, the could skip tedious back-and-forth conversations and implement screens
    with more confidence.
    \item[The need for a shared understanding of components] - Designers should be aware that their
    structure of designs can, to a certain degree, mimic the structure of code. As component based
    development is the golden standard of web development, both parties should share certain
    principles and best practices for components.
    \item[The need for more frequent communication] - Communication is key for a frictionless
    collaboration and results in great products. As stated in the book Lean UX, the user experience
    is "created by a team, not an individual user interface designer".
    \directcite[63]{gothelfLeanUXProduktentwicklung2016} As remote working models become more
    popular, this becomes an even bigger challenge. 
    \item[The need for common best practices] - Knowing when and why to use which tools and
    techniques would benefit the design handoffs drastically. Even more so, avoiding bad practices
    could save hours of reworking documentation, redoing designs or code implementations.
\end{description}

These four needs should be satisfied by the solution. The following sentence should serve as a
vision for the practical project.

"A solution that bridges disciplines, ... % Something fancy incorporating all the needs specified above

\subsubsection{The Idea of a Shared Interface}

\subsubsection{The System Architecture}

\subsubsection{The Design Language}