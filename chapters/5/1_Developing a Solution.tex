\newpage
\subsection{Developing a Solution}
Creating something that tailors to the struggles of both designers and engineers requires a
carefully considered solution. Guided by a clear vision, an innovative idea, a solid system and a
flexible design language were developed.

\subsubsection{The Vision of High Quality Collaboration}
Before starting to think about solutions, the actual needs of the two parties have to be examined
and a clear vision needs to be set. To do this, the data of the survey done beforehand was used as a
foundation.

\begin{description}
    \item[The need for cross-disciplinary knowledge] - Both designers and developers would benefit
          from more knowledge about the other discipline at places where it matters most. If designers
          knew why using variables for design tokens or why applying Figma Auto-Layout in frames
          matters to devs, they could use these techniques more intentionally. Similarly, if
          developers knew how to use design tools to their advantage, they could skip tedious
          back-and-forth conversations and implement screens with more confidence.
    \item[The need for a shared understanding of components] - Designers should be aware that their
          structure of designs can, to a certain degree, mimic the structure of code. As a component based
          approach is the golden standard of web development and found its way into interface
          design, both parties should share certain principles and best practices for components. 
    \item[The need for more frequent communication] - Communication is key for frictionless
          collaboration and results in great products. As stated in the book Lean UX, the user experience
          is "created by a team, not an individual user interface designer".
          \directcite[63]{gothelfLeanUXProduktentwicklung2016} As remote working models become more
          popular, this becomes an even bigger challenge.
    \item[The need for common best practices] - Knowing when and why to use which tools and
          techniques would benefit the design handoffs drastically. Even more so, avoiding bad practices
          could save hours of reworking documentation, redoing designs or code implementations.
\end{description}

These four needs should be satisfied by the solution. The following sentence should serve as a
vision for the practical project.

% Something fancy incorporating all the needs specified above
"A solution that bridges disciplines, builds shared understanding, educates all parties and
encourages communication"

\subsubsection{The Idea of a Shared Interface}
% Saying that the idea of an API is that no matter the platform there is a place that can be
% referred to, to get data, to be informed. Probably could take a page and quickly quote the
% definition of an API

% This concept can be applied to my practical work: A
% centralized place to get helpful tips, how-tos, best practices, communication ideas no matter the
% discipline. The DesignAPI is born: Design Alignment, Principles, Interchange. Being able to refer
% to it more often than once.

In programming realms, there is a widely known and used concept called API. Michael Goodwin
describes it perfectly saying an "API, or application programming interface, is a set of rules or
protocols that enables software applications to communicate with each other to exchange data,
features and functionality." \directcite{michaelgoodwinWhatAPIApplication2024} Upon hearing about a
great software, many devs respond with "Does it have an API?". This is because APIs are a place to
refer and get back to when you need functionality or data. it is a standard.

When it comes to the UI design process and the inter disciplinary collaboration, there is no such
thing as an API. Upon hearing about a great way to structure the design of components, no designer
would reply with "Is there an API for that?". My practical work aims to take the concept of an API
and create a centralized place to get helpful tips, how-tos, best practices and communication ideas
no matter the discipline. 

The DesignAPI is born: Design Alignment, Principles, Interchange. It is a place designers as well as
developers can refer and get back to as soon as they need practical knowledge or a quick tip. It
serves as a standard of how design-dev collaboration should be done.

\subsubsection{The System Architecture}
% Explaining what the thing is actually made of: 
% 7 Categories, list each one and what it is about and opportunities for content to fit the needs as
% well: f.e. Figmafication: Opportunity to show devs why using Figma is beneficial.
% 
% For each category there is a set of cards and a set of so-called message templates
% - % TODO: Add graphics that's like a blueprint of a card showing what's on every card (like my poster
% P-OWL): Consist of a front and a back. Front invites with a title and illustration, back a
%   selection of two to five sections that provide quick info. 
        
%%%%    % TODO: Add graphics that's like a blueprint of a card showing what's on every card (like my poster
%%%%    % P-OWL)
%
%   Explain each section: How to? - Why? - For Devs. Optional: Pro-Tip, Resource

% - Message Templates: To fit modern communication channels like Slack, MS Teams or others, there
%   are templates with Meme character. Again for each category at least one. Each one has a unique
%   purpose and aims to invite frequent playful communication with all team members. 



\subsubsection{The Design Language}
% Design language decisions
% Design has to be comfortable and inviting, so that users like picking it up and coming back to it often.

% Visual design colorful glass morphism: It should communicate transparency between the disciplines.
% It also adds depth (depth of information). A faint glow shows that it is somehow powerful and worth knowing
% The almost abstract illustrations keeps the content keeps the content mysterious and interesting.
% (You want to look at the other side to see what it is about => invokes curiosity)

% Fonts and colors. Each category has its own color shades. This makes it easy to distinct between
% the categories at a glance. 
% The selection of Fonts: 
% - Inter for paragraphs (great and versatile UI Font, default font in Figma
%   => makes designers feel right at home)
% - JetBrains Mono for headers (a font specifically made for developers)
% 
% Taken from both worlds of the spectrum, these fonts were the best choice for this project.

% TODO: Chance for graphics showing different colors and stuff

% Voice and Tone:
% - To make the topic feel more casual all content is written in a rather informal way. 
