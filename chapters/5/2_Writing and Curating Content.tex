\newpage
\subsection{Curating and Writing Content}
% Content selection:
% - Mostly from questionnaire and own experience of web UI designer
% - Blog posts from designers and developers
% - Tutorials and articles from Figma itself
% From that a list of topics and their category was selected.
The selection of content and formulating best practices, tips and guidelines was the toughest
challenge of the DesignAPI. It is the core of it all. With the categories in mind, a list of topics
was created from three main sources:

\begin{description}
    \item[Questionnaire and Individual Experiences] - Most topics and content were sourced from the
          analysis of the questionnaire. The real-life experiences, challenges, workarounds and
          already tested best practices were a great foundation. Combined with my year of experience
          in interface design, it was ensured that they could help others.
    \item[Blog posts from known designers and developers] - Articles helped immensely to further
          make sure, the content on the cards is supported by experiences of people who know about
          the struggles of design and code collaboration.
    \item[Tutorials and articles from Figma] - Some cards had to be Figma specific. Figma has a
          dedicated series of articles that show how repeating problems are correctly solved with its
          software. (www.figma.com/best-practices/) A few of the ideas, where included and referenced in
          the content of the cards.
\end{description}


% Writing content:
% For each topic I wrote down everything in the tone of voice specified that popped into my head for
% the necessary sections and included resource references where necessary. After that, I noticed
% that many sections for many cards were simply too long to fit on tiny cards. Which forced me to
% shorten everything => Great idea since it should be short. I chose to use ChatGPT as a way of
% shortening my original content. By prompting it with my original sentences for each card content
% and asking to keep the voice and tone of it just shorter. The answers given were then reworked 
% once more. It really sped up the process, while still staying in full control of the content
% written. This is a small testament of how Large Language Models (LLMs) can be used as a tool to
% enhance efficiency, without losing quality. 
% 

For each topic, I wrote the ideas and content in the defined tone of voice, ensuring all necessary
sections were addressed and included resource references where necessary. It was until after this
process, that it became clear many entries exceeded the space on the playing card format. However,
this constraint reminded of the initial goal of keeping information precise and brief.

% TODO: Add graphic that shows this process one more time.
% My original long content => Short content from ChatGPT => finished content with my changes
To refine the content, ChatGPT was used as a tool to help condense the paragraphs. The prompts used,
were crafted to keep the original tone while shortening everything. The generated responses were
then manually revised again to fully support the intended message. This highly iterative approach
both sped up the writing process and showed the potential of Large Language Models (LLMs) to support
content creation workflows. It enhanced efficiency without losing quality or intent.

In chapter 6, all of the cards along with their content are showcased.

